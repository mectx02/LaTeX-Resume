% 
% Preamble
% 
\documentclass[class=article, crop=false]{standalone}


\setlength{\parskip}{8pt}

\newcommand{\email}{your\_email.address@domain.com}

%
% Document Content
%
\begin{document}

Hi.

This is a sample cover letter that I just wrote to demonstrate what this looks 
like in general. You may find that some other options may tend to work better 
for your personal tastes, and that's okay. Feel free to customize this to your 
needs.

Hopefully this project template is easy to understand how to use. In case of 
it not making sense, though, here's a quick rundown:
\begin{enumerate}
    \item The cover letter and resume .tex files should not be edited --- those 
          are mostly set and forget. Various sections are edited separately in 
          the \verb|sections| folder and then imported into the major document.
          I would recommend against touching the \verb|cover.tex| file since 
          it's highly unlikely that you would need to change something in there.
          \verb|resume.tex| does need a little more customization in the sense 
          that you can add/delete sections that you feel are relevant/irrelevant 
          to you.
    
    \item You are free to create/delete/manipulate the importable sections as you 
          see fit. I've left the defaults pretty blank so that you can see how 
          a recommended styling should be done, but you can use whatever styling
          you desire.

    \begin{itemize}
         \item There's no \verb|.cls| file - everything is defined in standard 
          \verb|.tex| files and then imported into the two documents. Yes, this 
          means that there's more for you to do when customizing the items, but 
          I think it makes for a simpler project and easier to understand 
          interface for document creation.
    \end{itemize}

    \item When you're ready to compile a document, call \verb|make|, and it 
          should create three different pdf files. If it doesn't, you might need
          to install a utility called \verb|pdftk|, which is used to concatenate 
          the resume and cover letter file together.

    \begin{itemize}
        \item In the makefile, there's a variable called \verb|NAME| that you 
        will need to edit in order to rename the files to your name. It's 
        located at the very top of the makefile - just put your normal name in 
        there and \verb|make| should take care of the rest.
    \end{itemize}
\end{enumerate}

Hopefully that's all you should really need to know. 

Below is a sample finisher as an example.



% Save this part at the end for a nice finish
\vspace{5mm}

Sincerely, 

Jane Q. Taxpayer

\href{mailto:\email}{\email}

(555) 555-1234


\end{document}